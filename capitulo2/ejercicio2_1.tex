\subsection*{2.1 Discuta el impacto de las interfaces de usuario sobreconfiable}

\textbf{Lizbeth Espinoza}

\begin{quote}
	El usuario dependerá del software si es confiable. Ya que la aplicación es una herramienta necesaria para realizar el trabajo que el usuario requiere en su momento.
\end{quote}

\textbf{Emmanuel Vel\'asque Mtz}

\begin{quote}
	Muchas de las veces los errores son totalmente inebitables pero lejos de ser sorprendidos por esos errores los usuarios prefieren simplemente esperarar a que estos se solucionen, mientras que con otros tipos de servicios los clientes obtienen una gantia contra estas fallas.
\end{quote}
	
\textbf{Maria del Rosario Torres Saucedo}
\begin{quote}
La interfaces de usuario nunca son las esperadas debidos, en efecto las especificaciones son un modelo de lo que los usuarios buscan, pero, éste puede ser o no preciso ante los requerimientos del cliente actual.
\end{quote}
	
\textbf{Zaira Nayely Rincón Rodríguez}
\begin{quote}
Los productos de Software son comúnmente lanzados con una lista de "Bugs conocidos". Los usuarios de Software toman esto como la garantía de que la primera versión del producto es "buggy".
\end{quote}

\textbf{Isabel Cisneros López}
\begin{quote}
Cualquier desviación de la calidad conlleva a un sistema incorrecto, sín importar qué tan pequeño o seria sea la consecuencia de la desviación.
\end{quote}

\textbf{Mariel Catalina García Pacheco}
\begin{quote}
Si la consecuencia de error del software no es tan serio, el software inncorrecot podría llegar a ser confiable.
\end{quote}

